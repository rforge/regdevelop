
\documentclass[11pt]{article}
\usepackage{texab}
\usepackage{a4}

\def\note#1{\\ *** \emph{#1}\\}
\def\T{\texttt}
\def\code{\texttt}

\begin{document}

\section{Programming notes}
\Itm
regr: handling of data, subset, weights, ...
is subtile: weights may need to be evaluated before subset is taken.\\
The elimination of unused levels might be deferred to model.frame using the
argument \T{drop.unused.levels} -- 
but some fitting functions (at least \T{survreg}) do not pass the argument
on to it.\\
The component \T{\$allvars} of the \T{regr} result is good enough for 
\T{add1}. Therefore, the data argument is copied to the \T{\$funcall}
component of the result.

\Itm % nov 15
Binary factors: have a different name in drop1 than in coef.
\Arrow Use attr(\$x, "assign") and attr(\$terms,"term.labels")

\Itm
Factors created in the formula statement are difficult to handle
by plresx. They get a flag as \T{is.fac==2}, and fitcomp needs to know
that.
(Consider such strange terms as \T{factor(2*stelle), factor(pmin(a,b))})

\Itm
A namespace catch:\\
I export \T{Surv, strata, cluster} which come from the package
\T{survreg}, since otherwise, they cannot be used in formulas
unless \T{survreg} is loaded.

%% =======================================================================
\section{The Function \T{regr}}
\Tit{Arguments to regr}
\Itm
\T{family} % and \T{dist}
\T{normal = gaussian, binomial, poisson, gamma, cumlogit, multinomial,
weibull, lognormal, logistic, loglogistic, extreme, t}\\
\note{add cumloglog, ... for polr}
depends on type of response, which may be
\T{numeric, nmatrix, binary, bincount, ordered, factor}
\note{bincount defined by function Bincount}
\T{survival}: default ``family'' is \T{ph}, but others are available,
\T{"weibull"},...\\
Attribute \T{distribution} of y sets default for \T{family}\\
\T{Tobit} entails \T{family="gaussian"} as a default

\Itm
\T{method} !!! changed
\T{lm, rlm, nls, glm, polr, multinom, survreg, coxph}
\note{coxph needs definition of residuals}

\Tit{Value of regr}
Term table:
Coefficients for terms with a single coefficient.
Zero degrees of freedom in some case of interactions: continuous times
factor (obtained in survreg)


%% =================================================================
\section{Contrasts}
contr.wsum

contr.wpoly: orthogonalize wrt the design, not the scores.
$-> X^TX=$diag = $S^T W S$, if the rows of $S$ correspond to the scores.

Binary variables, whether 0-1, logical, or 2 level factors,
should get treatment contrasts by default for easy interpretation 
of the coefficient as well as coefficients of interactions.
But may also be centered (wrt the data.frame, not at 0.5);
which should be obtained either from contr.wpoly or by a tiny separate
function that does the same (to be done...).

%% =================================================================
\section{drop1, add1, step}
\Tit{NAs and subset}
NAs 
\Itm in resppmse \Arrow shorter object\$residual
\Itm in starting model \Arrow same
\Itm in add1 scope 

\Tit{step}
is modified. Why???

\Tit{drop1 and survreg}
\T{object\$df} is \T{length(coef)+1} instead of
\T{length(coef)-intercept}
%% =================================================================
\section{Residuals}
\Tit{Component Effects}
with interactions ?

\Tit{Fuzzy Residuals}
conditional distribution

%% =================================================================
\section{Residual Plots}
\Tit{QQ-plots}
not for glm. adequate distribution! (make sure for Gamma, weibull, ...!)

Conditional: show segments ony if conditional probability in range given by 
\T{condprobrange}

%% =================================================================
\section{Smooths}
Functions:

\code{smoothMM} calls \code{smoothM} calls \code{smooth}, 
which is \code{smoothRegr} by default.
\code{smoothMM} is called from \code{i.plotlws} or, in the case of
resdiduals from smooth (for qq and TAscale), by \code{plot.regr}.
\Itm
\code{smoothRegr} is essentially \code{loess}, with suitable parameter and
error handling. Returns NULL if less than 8 observations are provided
\Itm
\code{smoothMM}: smooth for multiple x and y, needed for multivariate
regression.
Yields a list of lists
\Itm
\code{smoothM}: smooth for multiple y, generating smooths by group. 
Yields a list.


%% =================================================================
\section{Documentation of Output}
tit and doc.

stamp

%% ==================================================================
\section{Miscellaneous}
\Itm
bookkeeping in lm

\$x = model.matrix, includes\\
  colnames= names of coef\\
  attr(,"assign") zuordnung zu term.labels, die aber nicht da sind\\
  attr(,"contrasts") for factors

\Itm
options ... mgp, 

\Itm
draw unimportant, extended items (reference lines, smooths) first



\end{document}



%%% Local Variables: 
%%% mode: latex
%%% TeX-master: t
%%% End: 

% -*- Mode: noweb; noweb-default-code-mode: R-mode; -*-
%\SweaveUTF8
\documentclass[11pt]{article}
\usepackage{graphicx}
\usepackage{Sweave}
\usepackage[utf8]{inputenc}
%% \usepackage{germanU}
%%- \usepackage[noae]{Sweave}
\usepackage[a4paper, text={14.5cm,22cm}]{geometry}
\usepackage{color} %uncomment BF
\usepackage{booktabs} % nice tables with \toprule \middlerule \bottomrule
\usepackage{amsmath} % for align
% \usepackage{wasysym} % for promille sign
% \usepackage{amssymb}
% \usepackage[textfont=it,font=small,labelfont=it]{caption}
\interfootnotelinepenalty=10000 % prevent LaTex from two-sided footnotes
\usepackage{regr-desc}

\addtolength{\textwidth}{2.5cm}%%--- 15.0 + 2.5 = 17.5
\addtolength{\oddsidemargin}{-1.04cm}

%% ================================================================

\begin{document}
\Sconcordance{concordance:regr-description.tex:/scratch/users/stahel/R/regdevelop/pkg/regr0/vignettes/regr-description.Rnw:%
1 45 1 1 7 86 1 1 5 29 0 1 3 81 1 1 2 4 0 1 3 28 0 1 2 103 1 1 2 3 0 1 %
4 12 0 1 2 16 1 1 2 15 0 1 3 4 0 1 2 100 1 1 4 5 0 1 3 204 1 1 6 5 0 1 %
3 94 1 1 6 5 0 1 3 76 1 1 2 5 0 1 3 21 1 1 2 1 3 13 1 2 3 99 1}

\setkeys{Gin}{width=1\textwidth}
\baselineskip 15pt

\title{\vspace*{-10mm}
The R-Function \T{regr} and Package \T{regr0} for an Augmented 
Regression Analysis}
\author{Werner A. Stahel, ETH Zurich}
\maketitle

\begin{abstract}\noindent
The R function \T{regr} is a wrapper function that allows for fitting
several different types of regression models by a unified call, and
provides more informative numerical and graphical output than the 
traditional \T{summary} and \T{plot} methods.
The package \T{regr0} contains the functions that go along with 
\T{regr} and a number of others.
It is written to make data analysis more effective by providing
user-oriented, flexible functions.
It is available from \T{R-forge} and is still in development.
\end{abstract}

\section{Introduction}

Regression models are fitted in the statistical programming environment R 
by diverse functions, depending on the particular type of model.
Outputs obtained by calling the function \T{summary} on the object produced
by the fitting function look often quite similar. Graphical output for
residual analysis is obtained by calling \T{plot}, but the result is not
always informative. 

The function \T{regr} allows for fitting various regression models
with the same statement, provides a more informative numerical output
and enhanced plots for residual analysis. 

\T{regr} proceeds by 
checking arguments, then calling the suitable fitting method from standard
R or other packages,
collecting useful statistics from the resulting object and a call of 
\T{summary} on it and adding a few to generate an object of class
\T{regr}. 
%%- The printing method for these objects shows the results that are usually 
%%- of interest.
%%- The plotting method produces a much more complete set of residual plots
%%- than the plotting methods for the usual fitting objects do.


In particular, the following models can be fitted by \T{regr}:
\begin{itemize}
\item 
  ordinary linear models, using Least Squares or robust estimation,
  by calling \T{lm} or \T{lmrob} from the \T{robustbase} package,
\item
  generalized linear models, by calling \T{glm},
\item
  multinomial response models, by calling \T{multinom} of package
  \T{nnet},
\item
  ordered response models, by calling \T{polr} of package
  \T{MASS},
\item
  models for survival data and Tobit regression, by calling
  \T{survreg} or \T{coxph} of package \T{survival},
\item
  multivariate ordinary linear models, by calling \T{lm}
\item
  nonlinear models, by calling \T{nls}
\end{itemize}

This document presents the main features of the package \T{regr0}
and explains the ideas behind them. 
It gives no details about the functions. They can be found in
the help files.

The package is available from \T{R-forge}, e.g. by calling\\
\T{install.packages("regr0", repos="http://r-forge.r-project.org")}.\\
The reason why it is not on CRAN and why it is called \T{regr0} rather than 
\T{regr} is that the author is still developing additional features and
does not yet want to guarantee upward compatibility.
It also means that comments and suggestions are very welcome:
\T{stahel\@ stat.math.ethz.ch} %% !!! at

\section{Numerical Output}
The useful numerical output of fitting functions is usually obtained by
calling \T{summary} on the object produced by the fitting method.
This results, for most functions, in a table showing the estimated
regression coefficients, their standard errors, the value of a test
statistic (t or z or deviance) and, for the 
ordinary linear model, a p value for the tests for zero coefficients. 
It is followed by an overall summary, usually including a test for
the hypothesis that all coefficients are zero, and a standard deviation of
the error and coefficient of determination, if applicable.

If there are factors (qualitative explanatory variables) in the model, 
the coefficients are not always interpreted adequately, and the 
respective standard errors, t and p values are of little use and often
misinterpreted. 
On the other hand, the information whether a factor has a significant
effect is not available from the summary but has to be obtained by calling 
\T{drop1} on the fit object. 
(The function \T{anova}, which seems to be suited according to its
name, usually does not answer this question.)

This situation cannot be called user friendly.
The function \T{regr} is meant to provide the results that are needed
without having the user call different functions and select the 
output that is safe to be interpreted.

Here is a result of printing a \T{regr} object.
\begin{Schunk}
\begin{Soutput}
  Blasting for Tunnel Excavation 

Call:
regr(formula = logst(tremor) ~ location + log10(distance) + log10(charge), 
    data = d.blast)
Fitting function:  lm 

Terms:
                  coef df  ciLow ciHigh  R2.x signif p.value p.symb
(Intercept)      2.995  1  2.772  3.219     .      .       .       
location             .  7      .      . 0.102   3.48       0    ***
log10(distance) -1.506  1 -1.630 -1.382 0.477 -12.13       0    ***
log10(charge)    0.623  1  0.546  0.699 0.102   8.17       0    ***
---
Signif. codes:  0  ***  0.001  **  0.01  *  0.05  .  0.1     1  

St.dev.error:  0.141   on 352 degrees of freedom
Multiple R^2:  0.798    Adjusted R-squared: 0.793    AIC:  -1409.04 
F-statistic:    155   on 9 and 352 d.f.,  p.value: 1.55e-116 

Effects of factor levels:
$location
  loc1        loc2        loc3        loc4        loc5        loc6       
  -0.0422 *    0.1085 ***  0.0876 *** -0.2078 *** -0.0760 **   0.0282    
  loc7        loc8       
  -0.0563     -0.0408    
\end{Soutput}
\end{Schunk}

\subsection{Standard output for continuous explanatory variables}
The ``Terms:'' table characterizes the effects of the individual terms in
the model. For continuous explanatory variables (the last 2 lines in the
example) it shows:
\begin{description}
\item[\T{coef},] the estimated value of the coefficient;
\item[\T{df},] degrees of freedom, $=1$ for continuous variables;
\item[\T{ciLow, ciHigh},] the limits of the confidence interval;
\item[\T{R2.x},] the coefficient of determination for regressing 
  the explanatory variable in question on the other terms in the model.
  This is one of the wellknown collinearity diagnostics.
\item[\T{signif},] a significance measure that is $>1$ for estimated
  coefficients differing significantly from 0, see below for its
  definition;
\item[\T{p.value},] the p value for testing if the coefficient could be
  zero. 
\item[\T{p.symb},] the usual significance symbols.
\end{description}
In fact, 2 more columns are contained in \T{rr\$termtable}, but they are
not printed by default:
\begin{description}
\item[\T{se},] the standard errors of the estimated coefficients; 
\item[\T{stcoef},] the estimated standardized coefficient,
  defined as \T{coef} times the standard deviation of the explanatory
  variable, divided by the standard deviation of the response (if
  the response is continuous as assumed here), see below for its use;
\end{description}
\Tit{User options.}
The default for the columns to be printed is set by an option stored 
in the \T{UserOptions} list and controled by the function 
\T{userOption} in the same way that the usual options are
controled by the function \T{options}.

\Tit{Significance}
The usual \T{summary} output of fitting functions includes the 
t (or z) values of the coefficients as a column in the coefficients 
table. 
They are simply the ratios of the two preceding columns. 
Nevertheless, they provide a kind of strength of the significance of the
coefficients. The p value may also serve as such a measure, but it is less 
intuitive as it turns tiny for important variables, making comparisons
somewhat more difficult than t values. 
The significance of t values depends on the degrees of freedom, but
informed users will know that critical values are around $\pm2$, and they will
therefore informally compare t values to $\pm2$. 
Based on these considerations, we introduce a new measure of significance
here. 

The new significance measure is defined as
\[
  \T{signif} = \T{t value}\;/\; \T{critical value}
\]
where \T{critical value} is the critical value $q_{df}$ of the 
t distribution and depends on the degrees of freedom of the residuals. 
%%- The definition is applicable for continuous explanatory
%%- variables as well as for binary factors.
%%- For other factors, we will extend this definition below.

\Tit{Confidence Intervals.}
The standard errors provided by the usual \T{summary} tables allow for
calculating confidence intervals for continuous explanatory variables,
by the formula $\T{coef} \;\pm\; q_{df}\cdot\T{std.error}$.
By default, they are shown in the Terms table. 
If the table gets too wide, the confidence limits may be suppressed.
They can then be calculated as
%%- The formula based on \T{signif} is
\[
  \T{coef}\cdot\;(1\pm\;1/\T{signif}\;)
\]
This is slightly more complicated for a calculation in the mind than
\T{coef}$\pm 2$\T{se}, 
but the formula shows an additional interpretation of \T{signif} 
in terms of the confidence interval:
If the input variable were scaled such that the confidence interval had 
half width 1, then the estimate would be \T{signif} units away from zero.

\Tit{Standardized Coefficients}
Standardized coefficients are meant to allow for a comparison of the 
importance of explanatory variables that have different variances.
They are also stored in \T{rr\$termtable} as the column \T{stcoef}.

\begin{Schunk}
\begin{Soutput}
 [1] "coef"    "se"      "ciLow"   "ciHigh"  "df"      "testst"  "signif" 
 [8] "p.value" "p.symb"  "stcoef"  "R2.x"   
\end{Soutput}
\begin{Soutput}
  Blasting for Tunnel Excavation 

Call:
regr(formula = logst(tremor) ~ location + log10(distance) + log10(charge), 
    data = d.blast)
Fitting function:  lm 

Terms:
                  coef stcoef df signif p.symb
(Intercept)      2.995      .  1      .       
location             .      .  7   3.48    ***
log10(distance) -1.506 -0.790  1 -12.13    ***
log10(charge)    0.623  0.406  1   8.17    ***
---
Signif. codes:  0  ***  0.001  **  0.01  *  0.05  .  0.1     1  

St.dev.error:  0.141   on 352 degrees of freedom
Multiple R^2:  0.798    Adjusted R-squared: 0.793    AIC:  -1409.04 
F-statistic:    155   on 9 and 352 d.f.,  p.value: 1.55e-116 

Effects of factor levels:
$location
  loc1        loc2        loc3        loc4        loc5        loc6       
  -0.0422 *    0.1085 ***  0.0876 *** -0.2078 *** -0.0760 **   0.0282    
  loc7        loc8       
  -0.0563     -0.0408    
\end{Soutput}
\end{Schunk}
Each of them shows the effect on the response of increasing ``its'' 
carrier $X^{(j)}$ by one standard deviation, as a multiple 
of the response's standard deviation. 
This is often a more meaningful comparison of the relevance of the input
variables. 

Note, however, that increasing one $X^{(j)}$ without also changing
others may not be possible in a given application, and therefore, 
an interpretation of coefficients can always be tricky.
Furthermore, for binary input variables, increasing the variable by one
standard deviation is impossible, since an increase can only occur from 0
to 1, and therefore, the standardized coeffient is somewhat 
counter-intuitive in this case.

\subsection{Factors}
For factors with more than two levels, (\T{location} in the example), there
are several coefficients to be estimated. 
Their values depend on the scheme for generating the 
dummy variables characterizing the factor, which is determined 
by the \T{contrasts} option (or argument) in use.
We come back to this point below (``Contrasts'').

Note that for factors with only two levels, the problem does not arise,
since the single 
coefficient can be interpreted in the straightforward manner as 
for continuous explanatory variables. \T{regr} therefore treats binary
factors in the same way as continuous explanatory variables.

The test performed for factors with more than two levels, which is shown 
in the \T{Terms} table by the \T{p.value} entry, 
is the F test for the whole factor (hypothesis: all coefficients are 0). 
It is obtained by calling \T{drop1}.
The significance measure is defined as 
\[
  \T{signif} = \sqrt{\T{F value}\;/\;q_{df1,df2}}
\]
where $q_{df1,df2}$ is the critical value of the F distribution.
It reduces to the former one for binary factors.

The collinearity measure \T{R2.x} for factors is a formal generalization of 
\T{R2.x} for terms with one degree of freedom, determined by applying
the relationship with the ``variance inflation factor'',
\T{R2.x}$=1/(1-\mbox{vif})$ to the generalized vif. 
[More explanation planned.]

\Tit{All coefficients for factors.}
The usual contrast option \T{contrasts=\penalty-100%
"contr.treatment"} gives the coefficients of the dummy variables 
a clear interpretation: 
they estimate the difference of the response between level $k$ and 
level 1 for $k>1$.
Another popular setting is \T{contrasts=\penalty-100%
"contr.sum"}, for which the $k$th coefficient estimates the effect 
of the $k$th level in such a way that the sum of all coefficients is 0.
For this setting, the last of these effects is not given in the
vector of coefficients, \T{rr\$coef}.

In order to avoid ambiguities, the \T{regr} output lists the 
estimated effects for all levels of the factors after the term table.
Even though the interpretation of significance and confidence intervals of
the individual effects of the levels is delicate, \T{regr} objects include
a full table, similar to the term table explained above, for each factor.
By default, however, only the estimated effects ar shown, together with the
``star symbols'' for their significance.

\Tit{Contrasts.}
An advantage of \T{contr.sum} over the usual \T{contr.treatment}
contrasts is that it avoids the (often unconscientious) choice of a
reference level -- the first level -- and allows, for each level, 
to assess immediately how large its effect is as compared to an overall
average effect.
An even more important advantage appears when interactions between factors
are included in the model: 
The main effects of one factor, including its significance, 
may still be interpreted (with caution) as average effects over the
levels of the other factor.

The \T{contr.sum} setting is not well adapted to unbalanced factors, since
the unweighted sum of coefficients is forced to be 0.
This leads to large standard errors when one of the levels has a low 
frequency. 
The \T{regr0} package provides the option \T{contr.wsum} for which the
sum of coefficients weighted with the frequencies of the levels is zero.
This type of contrasts is the default in \T{regr0}.

\subsection{Model summary}
The last paragraph of the output gives the summary statistics.
For ordinary linear models, the estimated standard deviation or the error
term is given first. (It is labelled ``Standard error of residual'' in the 
\T{lm} output, which we would label a misnomer.)
The \T{Multiple R\^{}2} is given next, together with its ``adjusted''
version, followed by the overall F test for the model.

For generalized linear models, the deviance test for the model is given.
If applicable, a test for overdispersion based on residual deviance is also
added. 

\subsection{Model Comparisons}
When model development is part of the statistical analysis, it is useful to
compare the terms that occur in different models under consideration.
There is a function called \T{modelTable} that collects coefficients, 
p values, and other useful information, and a \T{format} and \T{print} 
method for showing the information in a useful way.

\begin{Schunk}
\begin{Soutput}
[1] "no"       "datetime" "device"   "charge"   "distance" "tremor"   "location"
\end{Soutput}

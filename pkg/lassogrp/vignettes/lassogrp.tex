% -*- Mode: noweb; noweb-default-code-mode: R-mode; -*-
%\SweaveUTF8
\documentclass[11pt]{article}
\usepackage{graphicx}
\usepackage{Sweave}
\usepackage[utf8]{inputenc}
%% \usepackage{germanU}
%%- \usepackage[noae]{Sweave}
\usepackage[a4paper, text={14.5cm,22cm}]{geometry}
\usepackage{color} %uncomment BF
\usepackage{booktabs} % nice tables with \toprule \middlerule \bottomrule
\usepackage{amsmath} % for align
% \usepackage{wasysym} % for promille sign
% \usepackage{amssymb}
% \usepackage[textfont=it,font=small,labelfont=it]{caption}
\interfootnotelinepenalty=10000 % prevent LaTex from two-sided footnotes
\usepackage{regr-desc}

\addtolength{\textwidth}{2.5cm}%%--- 15.0 + 2.5 = 17.5
\addtolength{\oddsidemargin}{-1.04cm}

%% ================================================================

\begin{document}
\Sconcordance{concordance:lassogrp.tex:/scratch/users/stahel/R/regdevelop/pkg/lassogrp/vignettes/lassogrp.Rnw:%
1 42 1 1 5 15 1}

\setkeys{Gin}{width=1\textwidth}
\baselineskip 15pt

\title{\vspace*{-10mm}
R-Package \T{lassogrp} for the group lasso  method}
\author{Werner A. Stahel, ETH Zurich}
\maketitle

\begin{abstract}\noindent
  The lasso is a well known method for automatic model selection 
  in regression.
  This package is an extension of the \T{grplasso} package based on 
  the PhD thesis of Lukas Meier.
  It includes more user-oriented wrapper functions.
  
  This vignette waits to be written.
\end{abstract}

\section{Introduction}


{\small
This is the end of the story for the time being. I hope that you will
get into using \T{regr0} and have good success with your data analyses.
Feedback is highly appreciated.

Werner Stahel, \T{stahel at stat.math.ethz.ch}
}
\end{document}

%%% Local Variables: 
%%% mode: latex
%%% TeX-master: t
%%% End: 
